\chapter{Representação estudantil}
\DoPToC

\section{Diretório acadêmico}

O Diretório Acadêmico é uma entidade estudantil que representa, normalmente, os estudantes de um curso de nível superior dentro de uma Universidade. O D.A estabelece uma relação de compromisso com a classe estudantil do estabelecimento de ensino, uma vez que é formada pelos próprios acadêmicos. Porém sozinho o D.A. não consegue desenvolver suas atividades (que são voluntárias), para isso, precisa-se que todos os discentes, independentes de curso, turno e período, participem de forma ativa nos movimentos instigados pela referida entidade, que possui um objetivo elementar e crucial de proporcionar benefícios e lutar pelos direitos da classe, na luta constante por um ensino público, gratuito e de qualidade.

	Mas qual é a sua importância? É através da formação e existência ativa de um Diretório Acadêmico (D.A), que o estudante tem oficialmente voz dentro do estabelecimento de ensino, onde é a partir das articulações dos membros e a concessão direta da comunidade acadêmica, que permite o compartilhamento dos anseios, a busca por soluções e a elevada probabilidade de eficácia nos resultados. A Entidade desenvolve ainda atividades que visem proporcionar entretenimento à comunidade acadêmica, eventos de cunho informativo com o intuito de elevar intelectualmente o estudante, dentre outras ações.

	No nosso caso, temos o Dacomp, ou seja, o Diretório Acadêmico de Computação, que representa todos os cursos lotados no DCC junto aos orgãos da UFBA. Atualmente, esses cursos são:
  \begin{itemize}
 \item Bacharelado em Ciência da Computação.
 \item Bacharelado em Sistemas de Informação.
 \item Licenciatura em Computação.
 \end{itemize}

	O DACOMP congrega todos os alunos dos cursos, portanto todos os alunos são membros do diretório.
Todo ano é realizado o processo eleitoral para a diretoria, sendo que as chapas devem, preferencialmente, ser compostas de alunos dos diversos semestres (se possível calouros). 
\\

 \textbf{Como faço parar entrar em contato com a Dacomp?}
     \begin{itemize}
     
  \item www.dacomp.dcc.ufba.br
 \item e-mail: dacomp@dcc.ufba.br
 \item Telefone: (071)3263-6321
 \item Endereço: Instituto de Matemática - Campus de Ondina - UFBA. Av. Ademar de Barros, Ondina Salvador - Bahia - CEP: 40170-110 (veja como chegar ao DAComp)\\
     \end{itemize}

\section{Diretório central dos estudantes}

O DCE é a entidade responsável por representar todo o corpo discente de uma Universidade. Na UFBA, o DCE foi registrado em cartório em 19 (dezenove) de janeiro de 1999, pelo Conselho de Entidades de Base.
 
 A eleição de seus membros é definida pelo Movimento Estudantil da instituição no qual está inserido e costuma se dar de forma direta. A composição da diretoria (ou coordenação) pode ser na forma majoritária ou na forma proporcional.
 
Assim como os mecanismos eleitorais, a atuação da entidade é definida pelo conjunto do movimento estudantil da instituição, sendo que suas áreas de atuação mais comuns dizem respeito aos interesses dos estudantes perante à administração da instituição superior, às questões de política educacional e de política nacional. Além disso, o DCE pode manter relações com outras entidades representativas dos estudantes, como a União Nacional dos Estudantes (UNE), União Estadual dos Estudantes (UEEs) ou a Coordenação Nacional de Luta dos Estudantes (CONLUTE).
 
  Assim como o Diretório Acadêmico, a importância de se participar do DCE é que através dele o estudante passa a ter voz. E também, todos os estudantes, não os de um único curso, são membros.
  
  O DCE UFBA tem como finalidade:
     \begin{itemize}
 \item Defender com coerência, justiça e ética os direitos e interesses do corpo discente nos vários setores da vida universitária;

 \item Contribuir para o aperfeiçoamento do ensino universitário, pesquisa e extensa, assim como para o desenvolvimento cultural e político dos estudantes da UFBA;

 \item Lutar por uma universidade pública, gratuita, de qualidade, laica e socialmente referenciada;

 \item Lutar pela estruturação do movimento estudantil em todos os seus níveis de atuação;

 \item Defender o projeto histórico socialista de sociedade;\\
    \end{itemize} 
