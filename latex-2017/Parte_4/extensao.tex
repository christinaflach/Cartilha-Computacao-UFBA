\chapter{Extensão Universitária}

O DCC oferece atualmente três atividades permanentes de extensão -- o Programa Onda Digital, o Programa de Ações Pedagógicas para Formação de Professores de Computação (PROFCOMP) e a Especialização Avançada em Sistemas Distribuídos. Além dessas atividades, executamos outras ações eventuais e sob demanda, inclusive cursos de extensão para organizações públicas e privadas. 
      
      Ao longo dos anos, docentes do DCC tem organizado eventos regionais, nacionais e internacionais, e participado de projetos de extensão voltados para a sociedade baiana. 
      
      Para obter maiores informações sobre os cursos ou outras atividades de extensão, envie uma mensagem para a Profa. Debora Abdalla, coordenadora de Atividades de Extensão do DCC. 

\section{Programa Onda Digital}

O Onda Digital é um programa de Extensão permanente e visa a participação ativa da universidade dentro da comunidade em prol da inclusão digital. O Programa abrange ações educativas, de desenvolvimento de recursos humanos e técnicos e de criação e uso de software livre voltados para viabilização e melhoria dos processos de inclusão digital.
      
Atividades do Programa: 
       
\subsection{Criação do Grupo Colméia} 
       O Colméia tem por objetivo de disseminar o conhecimento digital, sem fronteiras sociais e sem barreiras culturais. O grupo é formado por professores do DCC, alunos voluntários do curso de computação com apoio da InfoJr, DACOMP, pessoas do CPD-UFBA e membros do PSL-BA. O grupo atuou no convênio com a ONG Eletro cooperativa e ministrou um curso de Iniciação a Informática para jovens.
       
\subsection{Oficina Teoria e prática educacional em projetos de inclusão digital}
       
Esta oficina visa preparar instrutores de cursos de inclusão digital no planejamento, execução e avaliação desses cursos. A oficina aborda, de forma teórico-prática, aspectos didáticos e pedagógicos que influenciam o sucesso de cursos voltados para a inclusão digital. A oficina terá duração de 12 horas.  

\subsection{Programa de Ações Pedagógicas para Formação de Professores de Computação (PROFCOMP)}

O PROFCOMP é um programa de Extensão permanente que integra ações educativas voltadas à inserção escolar do licenciando em computação como elemento mediatizador de ações pedagógicas interdisciplinares com uso de tecnologias digitais e do pensamento computacional com e sem uso de computadores. As ações do programa contam com apoio da Pró-Reitoria de Extensão da UFBA e pretendem contribuir à formação de professores e estudantes para a efetiva apropriação da "cultura digital" nas escolas. 

\section{Grupo de Programação (grupro)}

É um grupo de professores e alunos com vários interesses interligados, sendo o principal deles a participação em maratonas de programação. O grupro busca melhorar a qualidade dos cursos de graduação através da inserção da cultura de maratonas de programação no dia a dia das disciplinas de computação, aprimorar o perfil dos alunos egressos, e dar visibilidade a estes alunos através de bons desempenhos em competições relacionadas. Além disso, busca uma atuação mais representativa da cidade de Salvador e do estado da Bahia em competições nacionais e internacionais de programação. Informações em maratona.dcc.ufba.br 