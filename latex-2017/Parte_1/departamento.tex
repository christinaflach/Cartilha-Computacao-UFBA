\chapter{O Departamento de Ciência da Computação}
\DoPToC

\section{Apresentação}

O Departamento de Ciência da Computação (DCC) é um dos três departamentos do Instituto de Matemática e Estatística (IME) da Universidade Federal da Bahia (UFBA). O DCC, antigo Departamento de Processamento de Dados, foi institucionalizado pelo Decreto n$^{\mbox{o}}$. 62.241 de 8 de fevereiro de 1968. O DCC foi uma das primeiras instituições de ensino a oferecer um curso de graduação na área de Computação no país, o ``Bacharelado em Processamento de Dados''.

\section{Congregação}
\begin{itemize}
    \item Diretor da Unidade Universitária, que é seu presidente;
    \item Vice-Diretor da Unidade Universitária;
    \item Representantes da Unidade Universitária nos Conselhos Acadêmicos;
    \item Coordenadores dos Colegiados de Cursos de Graduação e de Programas de Pós-Graduação “stricto sensu”, sediados na Unidade;
    \item O Chefe de cada Departamento da Unidade;
    \item Um representante do corpo docente do quadro permanente lotado na Unidade;
    \item Um representante do corpo técnico-administrativo do quadro permanente lotado na Unidade;
    \item Representante do corpo discente na forma da lei.
\end{itemize}

\section{Colegiados}
\index{Colegiados}

Os colegiados dos cursos de graduação e programas de pós-graduação são responsáveis pela coordenação e supervisão das atividades dos cursos, pela fixação de diretrizes e orientações pedagógicas para o respectivo curso de graduação e de pós-graduação.

\section{Centro de Apoio Administrativo}
\index{CEAD}

"Denominado CEAD, tem como missão assessorar o Dirigente da Unidade no que se refere ao planejamento e administração da Unidade".\\

\section{Centro de Atendimento à Graduação}
\index{CEAG}

"Denominado CEAG, tem como missão assessorar administrativamente todas as atividades que visem a manutenção e o desenvolvimento do ensino de graduação do Instituto de Matemática e Estatística".\\

\section{Centro de Atendimento à Pós-Graduação}
\index{CEAPG}

"Denominado CEAPG, tem como missão assessorar administrativamente todas as atividades que visem a manutenção e o desenvolvimento dos programas de pós-graduação do Instituto de Matemática e Estatística".\\
\begin{itemize}
\item Para mais informações visite: http://www.im.ufba.br 
    
\end{itemize}


\section{Estrutura física}
O curso de Bacharelado em Ciência da Computação da UFBA tem seu coração no instituto de Matemática e Estatística, podendo também utilizar também apêndices como O PAF1 onde ocorrem aulas de diversas matérias no ramo da matemática e da computação. Entretanto é no IME onde o universitário encontra os laboratórios onde pode tanto ter suas aulas quanto utilizar para fazer atividades relacionadas ao curso. 
\begin{itemize}
\item Mas como acessar os computadores de um dos laboratórios abertos para estudantes?
É simples, O aluno recebe e-mail contendo um usuário e uma senha inicializados pelo GRACO no início do semestre (que pode ser alterada pelo usuário) e assim tem acesso aos computadores do laboratório.
\item Onde?
Sala 147 no IME é um dos laboratórios abertos aos estudantes. No entanto, recomendamos procurar saber com seu professor(a) mais informações sobre outros laboratórios disponíveis!
\end{itemize}

