\chapter{Contexto histórico}
\DoPToC
\newpage
\section{UFBA - Quando tudo começou}
O ponto inicial da UFBA é a chegada da família real portuguesa ao Brasil em 1808, após a fugir da França comandada por Napoleão Bonaparte. Nesse mesmo ano, o atual regente, João VI, assina o documento que criar a Escola de cirurgia da Bahia, atualmente chamada de Faculdade de Medicina da Bahia.  Ao longo do século outros cursos foram criados os cursos de Farmácia (1832) e Odontologia (1864), a Academia de Belas Artes (1877), Direito (1891) e Politécnica (1896).

Entretanto, até 1946 existiam apenas diversas faculdades separadas, mas nesse ano, através do Decreto-Lei 9155, as Unidades de Ensino Superiores já existentes deverão ser incorporadas a Universidade da Bahia (Antigo nome da UFBA). Sendo composta inicialmente pelos seguintes estabelecimentos de ensino superior: Faculdade de Medicina da Bahia e suas escolas anexas (Odontologia e Farmácia), Faculdade de Direito da Bahia, Escola Politécnica da Bahia,Faculdade de Filosofia da Bahia e Faculdade de Ciências Econômicas. Tendo como o primeiro reitor Edgar Santos. 
  
É somente em 1965 que, por meio da Lei n\textsuperscript{o} 4.759, a Universidade da Bahia passa a se chamar Universidade Federal da Bahia. Três anos depois, com o  Decreto Federal n\textsuperscript{o} 62.241, a UFBA é reestruturada e são criados os novos Institutos de Matemática, Física, Química, Biologia, Geociências e Ciências da Saúde, as Escolas de Biblioteconomia e Comunicação, e de Nutrição e a Faculdade de Educação. A partir dai passa a ser constituída de 24 unidades universitárias e cinco órgãos de administração superior.

\section{O Departamento de Ciência da Computação}
\index{DCC}

A UFBA foi uma das primeiras instituições de ensino a oferecer um curso de graduação, no ano de 1967,  Bacharelado em Processamento de Dados, na área de computação no país.Em 8 de fevereiro de 1968, com o Decreto n. 62.241, o Departamento de Ciência da Computação foi institucionalizado.
 
Em 1996 o curso de "Processamento de Dados" é  renomeado para "Ciência da Computação". No ano seguinte, começa-se a pensar em ampliar o DCC, em 1997 é apresentado um planejamento estratégico visando ampliar e qualificar o corpo docente do DCC ("Pessoas"), implantar pós-graduação, melhorar a gestão do DCC e de seus recursos computacionais ("Redes"), bem como diversas demandas -- por exemplo, espaço físico ("Infra).

Em 2010 foi integrado o curso de "Sistema de Informações", que embora seja uma área ainda recente e em grandes níveis de desenvolvimento, já conquistou um espaço relevante no mercado de trabalho. Ainda no mesmo ano, é criado o curso de "Licenciatura em Computação".
  
No ano de 2012, inicia-se as atividades do  MMCC - Mestrado Multiinstitucional em Ciência da Computação. Em 2014 é criado PGCOMP - Programa de Pós-graduação em Ciência da Computação, com as seguintes linhas de pesquisas: 
       \begin{itemize}

  \item Computação Visual e Sistemas Inteligentes
 \item Engenharia de Software
 \item Redes e Sistemas Distribuídos
 \item Teoria, Lógica e Métodos Formais
 \item Web e Banco de Dados\\
                 \end{itemize}
	Também há o Programa de Pós-Graduação em Mecatrônica com as linhas de pesquisas:

\begin{itemize}
  \item Integração da Manufatura
 \item Sistemas Computacionais\\
\end{itemize}
  