\chapter{Avaliações}
\DoPToC
\section{Avaliação de rendimento acadêmico}

O processo de avaliação de aprendizagem compreende a apuração da frequência às aulas e demais atividades acadêmicas e às notas atribuídas a essas atividades. A metodologia de ensino-avaliação será determinada pelo professor ou pelo grupo de professores do determinado componente curricular, de modo que o programa do componente seja respeitado.

\begin{itemize}
        \item O método de avaliação deverá ser divulgado até o final da segunda semana letiva;
        \item O professor ou grupo de professores responsável pela disciplina deve realizar pelo menos duas avaliações de aprendizagem;
        \item Nos componentes nos quais não cabe atribuição de notas às atividades acadêmicas, o resultado final será expresso por menção de aprovação ou reprovação, conforme o caso;
        \item Nos componentes nos quais cabe resultado numérico de avaliação, o resultado obedecerá a uma escala de zero (0) a dez (10), com uma casa decimal;
        \item Será considerado aprovado, em cada componente curricular, o aluno que cumprir a frequência mínima de 75\% (setenta e cinco por cento) às aulas e às atividades e obtiver nota final igual ou superior a cinco (5,0) ou aproveitamento nos componentes curriculares cujos resultados não forem expressos por nota;
        \item As atividades acadêmicas passíveis de avaliações deverão ser agendadas com, pelo menos, cinco (05) dias úteis de antecedência e, preferencialmente, figurar no plano de ensino do componente curricular, respeitados os dias e horários destinados ao ensino do mesmo;
        \item O resultado de cada avaliação deverá ser divulgado antes da realização da avaliação seguinte com, no mínimo, dois (02) dias úteis de antecedência;
\end{itemize}

O estudante tem o direito de solicitar a reavaliação da avaliação de aprendizagem, através de uma solicitação fundamentada pelo aluno e encaminhada ao Departamento ou equivalente, se requerida até três (03) dias úteis após a divulgação do resultado. Em primeira instância, essa será reavaliada pelo professor que a atribuiu, e em segunda instância, por uma comissão designada pelo Departamento ou equivalente, composta por três (03) professores, ouvido o professor responsável pela avaliação.

O aluno que faltar a qualquer das avaliações previstas terá direito à segunda chamada, se a requerer ao Departamento ou equivalente responsável pelo componente curricular, até cinco (05) dias úteis após a sua realização, comprovando-se uma das seguintes situações:
    \begin{itemize}
        \item Direito assegurado por legislação especifica;
        \item Motivo de saúde comprovado por atestado médico;
        \item Razão de força maior, julgado a critério do professor responsável pelo componente curricular.
    \end{itemize}
    A avaliação em segunda chamada será feita pelo próprio professor da turma, em horário por este designado com, pelo menos, três (03) dias de antecedência, consistindo do mesmo tipo de avaliação, com conteúdo similar ao da primeira chamada. A falta à segunda chamada implica em nota zero (0).
    
\section{Coeficiente de rendimento} \index{Coeficiente de rendimento}

O coeficiente de rendimento é o índice que indica em números o seu desempenho acadêmico como estudante. É calculado tomando como base as nota obtidas nas disciplinas e as cargas horárias das mesmas.
\begin{itemize}
    \item CR =  PHC / CH;
    \item PCH = Notas multiplicadas pelas cargas horárias dos componentes curriculares;
    \item CH = Carga horária total dos componentes matriculados.
\end{itemize}
Site para calcular seu Coeficiente de Rendimento: http://brinks.guisehn.com/teste.html.